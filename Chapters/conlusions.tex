\chapter*{Conclusions}
\markboth{CONCLUSIONS}{}
\addcontentsline{toc}{chapter}{Conclusions}
\label{chap:conclusions}
\thispagestyle{fancy}

In conclusion, this report explored various aspects of quantum information processing using superconducting qubits, as employed within the Quantum Device Lab. 
We provided an overview of superconducting qubits, with a particular focus on transmon qubits and a device used for the generation of entangled photons travelling through a waveguide.

In \cref{chap:crosstalk}, we addressed the issue of microwave signal crosstalk, frequently encountered in superconducting qubit systems.
Our approach, inspired by the work of Nuerbolati \emph{et al.} \cite{crosstalk}, aimed to mitigate crosstalk by measuring the AC Stark shift on a qubit induced by signals intended for other qubits through a Ramsey experiment.

Our investigation began with the detection of crosstalk between a target qubit and a control drive not directly capacitively coupled to the target qubit.
We estimated a crosstalk level of approximately $3.53\%$ between the gate line of the control qubit and the target qubit.

Subsequently, we characterized the compensation signal to be sent to the target qubit using flat top pulses to cancel the crosstalk.
Optimal compensation parameters were determined and the successful cancellation of crosstalk was verified through Ramsey and Rabi experiments.
We then repeated the process using Gaussian-shaped pulses and once again, successful crosstalk cancellation was confirmed using Ramsey and Rabi experiments.

In \cref{chap:2_qubit_gates}, we developed a model for the interaction between sidebands of the entangled system of two qubits. 
These results were employed to calibrate and determine the optimal frequency and duration of interaction between the two qubits, in order to implement two-qubit gates between storage-storage and storage-emitter qubits.

Finally, in \cref{Chap:graph_states}, we introduced graph states and their significance in quantum computation and communication.
Specifically, we introduced a particular type of graph state: the tree graph state.
We elucidated how this particular graph state can be generated using our current chip and its importance for applications such as implementing a photon-loss-tolerant one-way quantum repeater.
