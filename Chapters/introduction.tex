\chapter*{Introduction}
\markboth{INTRODUCTION}{}
\addcontentsline{toc}{chapter}{Introduction}
\label{chap:intro}

\pagenumbering{arabic}
\pagestyle{fancy}
\thispagestyle{fancy}

In the ever-evolving landscape of quantum information processing, the emergence of quantum computers has sparked immense excitement and curiosity among scientists, researchers, and technologists worldwide.
Theoretical physicist Richard Feynman brought attention to this insight in the early 1980s.
He notably argued that classical computers faced inefficiencies when simulating quantum mechanics \cite{Feynman}. 
Drawing from the groundbreaking contributions of Ed Fredkin and Tomasso Toffoli \cite{Ed_tommy}, Feynman introduced an alternative computational model that exploited the fundamental principles of quantum mechanics.

Unlike classical computers that rely on bits to represent information as either 0 or 1, quantum computers leverage the principles of quantum mechanics to manipulate quantum bits or \emph{qubits}.
One of the most intriguing aspects of quantum computing is its potential to revolutionize fields ranging from cryptography and drug discovery to optimization problems and artificial intelligence.

In addition to quantum computing, quantum communication \cite{Quantum_communication} represents another frontier in quantum technology that holds tremendous promise for secure and efficient data transmission.
Entangled photons, central in quantum communication, enable protocols like quantum teleportation \cite{Quantum_teleportation} and quantum cryptography. 
Exploiting the entanglement of photon pairs, researchers have demonstrated secure information transmission over extended distances, laying the foundation for quantum-secured communication networks.

In our research team, we aim to generate entangled photons in states as the tensor-network states \cite{tensor_networks}.
Utilizing a chip equipped with four transmon superconducting qubits, we allocate two qubits for the creation and entanglement of matter-based qubits, while the remaining two are for emitting the qubit states into the emission line as flying photons, as it will be discussed in \cref{Chap:experimental}. 
This approach enables us to successfully generate 2D cluster states \cite{HernandezAnton2023}.
These states find applications across various domains, including measurement-based quantum computing (MBQC) \cite{MBQC, MBQC_3}, quantum communication \cite{Quantum_communication}, metrology \cite{metrology}.

In this project, we tackled the issue of microwave signal crosstalk \cite{Crosstalk_intro_1, crosstalk_intro_2, signal_crosstalk}, which is a common challenge in working with superconducting circuits.
Our aim was to eliminate this effect to achieve higher operation accuracy on our chip.
Starting from existing research \cite{crosstalk} and exploring our own ideas, we pursued various strategies to address this issue.
The strategy and results will be discussed in \cref{chap:crosstalk}.

Following this, in \cref{chap:2_qubit_gates} we focused on modeling interactions between pairs of qubits, enabled by tunable couplers.
This allowed us to fine-tune the calibration of interactions between qubit pairs, an important step in implementing two-qubit gates.

Finally, in \cref{Chap:graph_states} we investigated the feasibility of implementing a specific type of photonic graph state: the \emph{tree graph state}.
These graph-state-based quantum repeaters \cite{Why_tree_graph_state} offer a promising solution by utilizing quantum encoding to mitigate challenges such as photon losses and operational errors, thereby reducing the reliance on extensive quantum memory coherence times.
We find a protocol to generate such graph states using one of our existing devices.